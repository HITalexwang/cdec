  \begin{tabular}{c|p{\mmsz}|l}
    \textbf{$z$} & \textbf{Description} & \textbf{Examples} \\
                 &                      & format: $c_{-1}$ \p\ $c_{+1}$  \gloss{translation} \\ \hline
    0 & \parbox{0.4\textwidth}
        {75\% concerns the infix +s+, both as \p\ and as $c_{+1}$} &
        \begin{minipage}{0.4\textwidth}
          \begin{tabular}{ll} 
            de \pemph{europe+} +s+ & \gloss{the european+} \\
            europe+ \pemph{+s+} +e & \gloss{european}
          \end{tabular}
        \end{minipage} \\ \hline

    1 & \parbox{0.4\textwidth}
        {$>$85\% prefix morphemes as \p, mostly noun stems with plural forming $c_{+1}$.} &
        \begin{minipage}{0.4\textwidth}
          \begin{tabular}{ll} 
            ... \pemph{resolutie+} +s & \gloss{resolutions} \\
            ... \pemph{kilometer+} +s & \gloss{kilometres} \\
          \end{tabular}
        \end{minipage} \\ \hline

    4 & \parbox{0.4\textwidth}
        {$>$99\% verb stems as \p} &
        \begin{minipage}{0.4\textwidth}
          \begin{tabular}{ll} 
            ge+ \pemph{+maakt} ... & \gloss{made} \\
            ver+ \pemph{+werpt} ...& \gloss{reject(s) [verb]} \\
            samen+ \pemph{+brengt}...& \gloss{bring together} \\
          \end{tabular}
        \end{minipage} \\ \hline

    5 & \parbox{0.4\textwidth}
        {25\% \p-instances are the prefix ``ver+''. \\ 
         25\% \p-instances are the domain-specific suffix ``+missie''. \\ 
         14\% $c_{+1}$-instances are the adjective-deriving suffix ``+isch''.} &
        \begin{minipage}{0.4\textwidth}
          \begin{tabular}{ll} 
            ... \pemph{ver+} +slag & \gloss{report} \\
            ... \pemph{ver+} +beter & \gloss{improve} \\
            com+ \pemph{+missie} ... & \gloss{commission} \\
            ?? isch \\
          \end{tabular}
        \end{minipage} \\ \hline

    6 & \parbox{0.4\textwidth}
        {$>$99\% adjective stems, followed by a marker for definiteness.} &
        \begin{minipage}{0.4\textwidth}
          \begin{tabular}{ll} 
            ...\pemph{interessant+} +e & \gloss{interesting} \\
            ...\pemph{etisch+} +e & \gloss{ethical} \\
          \end{tabular}
        \end{minipage} \\ \hline

    7 & \parbox{0.4\textwidth}
        {$\pm$13\% are stems followed by ``+elijk'', often forming adverbs(?). \\
         $\pm$17\% are noun-deriving suffixes ``+heid''/``+heden''.} &
        \begin{minipage}{0.4\textwidth}
          \begin{tabular}{ll} 
            ...\pemph{aanvank+}  +elijk & \gloss{initially} \\
            ... \pemph{begrijp+}  +elijk & \gloss{understandably} \\
            vrij+  \pemph{+heden}  ... & \gloss{liberties} \\
            bevoegd+ \pemph{+heid}  ... & \gloss{authorisation} \\
          \end{tabular}
        \end{minipage} \\ \hline

    10 & \parbox{0.4\textwidth}
        {72\% full words, mostly (compound) nouns or things acting as such. Where $c_{-1}$ is an article, nouns are almost exclusively neuter singular.} &
        \begin{minipage}{0.4\textwidth}
          \begin{tabular}{ll}
            het \pemph{uit+breken}  van & \gloss{the outbreak of} \\
            ... \pemph{drie+jaren+plan}  ... & \gloss{three-year plan} \\
          \end{tabular}
        \end{minipage} \\ \hline

    16 & \parbox{0.4\textwidth}
        {$>$92\% full words, similar to but much larger than Cat. 10. Where $c_{-1}$ is an article, nouns are almost exclusively plural or male/female singular.} &
        \begin{minipage}{0.4\textwidth}
          \begin{tabular}{ll}
            de \pemph{wet+geving}  van & \gloss{the legislation of} \\
            deze  \pemph{wij+zig+ing}  ... & \gloss{this modification} \\
          \end{tabular}
        \end{minipage} \\ \hline
  \end{tabular}

